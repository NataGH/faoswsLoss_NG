%\VignetteIndexEntry{Statistical Working Paper on Imputation Methodology for the FAOSTAT Production Domain}
%\VignetteEngine{knitr::knitr}
\documentclass[nojss]{jss}\usepackage[]{graphicx}\usepackage[]{color}
%% maxwidth is the original width if it is less than linewidth
%% otherwise use linewidth (to make sure the graphics do not exceed the margin)
\makeatletter
\def\maxwidth{ %
  \ifdim\Gin@nat@width>\linewidth
    \linewidth
  \else
    \Gin@nat@width
  \fi
}
\makeatother

\definecolor{fgcolor}{rgb}{0.345, 0.345, 0.345}
\newcommand{\hlnum}[1]{\textcolor[rgb]{0.686,0.059,0.569}{#1}}%
\newcommand{\hlstr}[1]{\textcolor[rgb]{0.192,0.494,0.8}{#1}}%
\newcommand{\hlcom}[1]{\textcolor[rgb]{0.678,0.584,0.686}{\textit{#1}}}%
\newcommand{\hlopt}[1]{\textcolor[rgb]{0,0,0}{#1}}%
\newcommand{\hlstd}[1]{\textcolor[rgb]{0.345,0.345,0.345}{#1}}%
\newcommand{\hlkwa}[1]{\textcolor[rgb]{0.161,0.373,0.58}{\textbf{#1}}}%
\newcommand{\hlkwb}[1]{\textcolor[rgb]{0.69,0.353,0.396}{#1}}%
\newcommand{\hlkwc}[1]{\textcolor[rgb]{0.333,0.667,0.333}{#1}}%
\newcommand{\hlkwd}[1]{\textcolor[rgb]{0.737,0.353,0.396}{\textbf{#1}}}%

\usepackage{framed}
\makeatletter
\newenvironment{kframe}{%
 \def\at@end@of@kframe{}%
 \ifinner\ifhmode%
  \def\at@end@of@kframe{\end{minipage}}%
  \begin{minipage}{\columnwidth}%
 \fi\fi%
 \def\FrameCommand##1{\hskip\@totalleftmargin \hskip-\fboxsep
 \colorbox{shadecolor}{##1}\hskip-\fboxsep
     % There is no \\@totalrightmargin, so:
     \hskip-\linewidth \hskip-\@totalleftmargin \hskip\columnwidth}%
 \MakeFramed {\advance\hsize-\width
   \@totalleftmargin\z@ \linewidth\hsize
   \@setminipage}}%
 {\par\unskip\endMakeFramed%
 \at@end@of@kframe}
\makeatother

\definecolor{shadecolor}{rgb}{.97, .97, .97}
\definecolor{messagecolor}{rgb}{0, 0, 0}
\definecolor{warningcolor}{rgb}{1, 0, 1}
\definecolor{errorcolor}{rgb}{1, 0, 0}
\newenvironment{knitrout}{}{} % an empty environment to be redefined in TeX

\usepackage{alltt}
\usepackage{url}
\usepackage[sc]{mathpazo}
\usepackage{geometry}
\geometry{verbose,tmargin=2.5cm,bmargin=2.5cm,lmargin=2.5cm,rmargin=2.5cm}
\setcounter{secnumdepth}{2}
\setcounter{tocdepth}{2}
\usepackage{breakurl}
\usepackage{hyperref}
\usepackage[ruled, vlined]{algorithm2e}
\usepackage{mathtools}
\usepackage{draftwatermark}
\usepackage{float}
\usepackage{placeins}
\usepackage{mathrsfs}
\usepackage{multirow}
%% \usepackage{mathbbm}
\DeclareMathOperator{\sgn}{sgn}
\DeclareMathOperator*{\argmax}{\arg\!\max}

\title{\bf Statistical Working Paper on Imputation Methodology for the
  FAOSTAT Production Domain}

\author{Joshua M. Browning and Michael C. J. Kao\\ Food and Agriculture Organization \\ of the United Nations}

\Plainauthor{Joshua M. Browning and Michael C. J. Kao}

\Plaintitle{Statistical Working Paper on Imputation Methodology for
  the FAOSTAT Production Domain}

\Shorttitle{Imputation Methodology}

\Abstract{ 

  This paper proposes a new imputation method for the FAOSTAT
  domains based on linear mixed model and ensemble learning.
  
  The proposal provides a resolution to many of the shortcomings of the
  current approach, and offers a flexible and robust framework to
  incorporate further information to improve performance.
    
  A detailed account of the methodologies is provided. The linear
  mixed model demonstrates an ability to capture cross-country and
  cross-commodity information; meanwhile, the ensemble learning approach
  provides flexiblity and robustness where traditional imputation methods
  of applying a single model may have failed.
    
}

\Keywords{Imputation, Linear Mixed Model, Ensemble Learning}
\Plainkeywords{Imputation, Linear Mixed Model, Ensemble Learning}

\Address{
  Joshua M. Browning and Michael C. J. Kao\\
  Economics and Social Statistics Division (ESS)\\
  Economic and Social Development Department (ES)\\
  Food and Agriculture Organization of the United Nations (FAO)\\
  Viale delle Terme di Caracalla 00153 Rome, Italy\\
  E-mail: \email{joshua.browning@fao.org, michael.kao@fao.org}\\
  URL: \url{https://svn.fao.org/projects/SWS/RModules/faoswsImputation/}
}
\IfFileExists{upquote.sty}{\usepackage{upquote}}{}
\begin{document}

\section*{Disclaimer}
This working paper should not be reported as representing the views of
the FAO. The views expressed in this working paper are those of the
author and do not necessarily represent those of the FAO or FAO
policy. Working papers describe research in progress by the author(s) and
are published to elicit comments and to further discussion.\\

It is in the view of the author that imputation should be implemented
as a last resort rather than as a replacement for data
collection. Imputation itself does not create information; it merely
create observations based on assumptions.\\

This paper is dynamically generated on \today{} and is subject to
changes and updates.

\section{Introduction}
Missing values are commonplace in the agricultural production domain,
stemming from non-response in surveys or a lack of capacity by the
reporting entity to provide measurement. Yet a consistent and
non-sparse production domain is of critical importance to Food Balance
Sheets (FBS), thus accurate and reliable imputation is essential and a
necessary requisite for continuing work. This paper addresses several
shortcomings of the current work and a new methodology is proposed in
order to resolve these issues and to increase the accuracy of
imputation.\\

The primary objective of imputation is to incorporate all
available and reliable information in order to provide best estimates of
food supply in FBS.\\

Presented in table \ref{tab:swsflag} is a description of the existing
flags in the current Statistical Working System (SWS). In this
exercise, estimated and previously imputed data are marked as
either \textbf{E} or \textbf{T} and are the target values to
be imputed.\\

\textbf{FIX THIS TABLE!!!}

\begin{table}[h!]
  \label{tab:swsflag}
  \caption{Description of the flags in the Statistical Working System}
  \begin{center}
    \begin{tabular}{|c||p{12cm}|}
      \hline
      Flags & Description\\
      \hline
      & Official data reported on FAO Questionnaires from countries\\
      / & Official data reported on FAO Questionnaires from countries\\
      * & Commodity International Organizations\\
      X & Commodity International Organizations\\
      P & Estimated data using trading partners database\\
      F & FAO estimate\\
      C & Calculated data\\
      B & Data obtained as balance\\
      T & Extrapolated/interpolated\\
      M & Not reported by country\\
      E & Expert sources from FAO (including other divisions)\\
      \hline
    \end{tabular}
  \end{center}  
\end{table}

\section{Exploratory Data Analysis}

\subsection{Yield Data}

\subsection{Production Data}

\subsection{Seed Data}

\subsection{Data Quality Issues}

During the development of the methodology, we have encountered several
data quality issues which required us to review and redefine our
initial methodology. These are not exceptions, rather they are
prevalent in the production domain and the analyst should bear in mind
these characteristics.

\subsubsection{Extremely High Sparsity}
Missing values are expected given that the goal of the task is
to impute missing values, but one may be stunned at the sparsity of
the data. For commodities such as pepper, merely 20\% of the data are
available; this raises the question whether imputation approaches are
even valid at all.

\subsubsection{Diverging Trends and shocks}
Another issue arose from the quality of the data reported and
recorded. It is not uncommon to observe unexplainable diverging trends
or shocks of production and area harvested which resulted in exploding
yield. The yield of Okra for Bahrain and Senegal in figure
\ref{fig:okra-yield-explore} are prime examples. Coconut production of
China in 2008 is another example: a change in classification resulted
in large escalation of production while the area harvested remained
similar to the previous year. This resulted in a three-fold increase in the
yield solely for that particular year.\\

\subsubsection{Distinguishing between zeroes, missing values and N/As}
The analyst should be made aware of the fact that, although a framework
does exist to distinguish zero and missing values in the database, in
practice this may not be the case.\\

These observations prompted us to devise a robust methodology to safeguard
ourselves from non-sensical imputation.\\

\section{Proposed Methodology}

The imputation of missing observations is traditionally done via a model.  For example, we may consider a simple global mean model where we compute the mean of all available observations and use that value to impute missing observations.  Alternatively, we could use a more complex model (i.e. linear/exponential/logistic regression, spline, etc.), fit the model to the available data, and then estimate missing values using this model.  However, this approach has two problems.  First, we may choose a poor model and thus obtain poor estimates.  Second, we have to specify which model to use for each set of data, and this could be very tedious if we have many time-series to impute.  To avoid this problems, we consider ensemble models.

\subsection{Ensemble Imputation}

Ensemble learning refers to the process of building a collection of
simple base models or learners which are later combined to obtain a
composite model or prediction.  One of the most famous applications of ensemble learning was the prediction of movie ratings held by Netflix in which the
top two performers both used an ensemble of different models.  Ensembles are very popular in the data-mining community because of their ability to combine multiple models and come up with an estimate that is better than any of the individual models.\\

The method consist of two steps:
\begin{enumerate}
  \item Building multiple models/learners.
  \item Combining the models or predictions.
\end{enumerate}

The ensemble method reduces the risk of choosing a poor model as we are averaging multiple models.  Thus we reduce the risk of implementing a single model which may produce poor imputations for a certain subset of data. Moreover, model selection is unecessary, since all model are included in the final ensemble.\\

Thomas Dietterich describes several problems in machine learning in his paper ``Ensemble Methods in Machine Learning'' (see \url{http://www.eecs.wsu.edu/~holder/courses/CptS570/fall07/papers/Dietterich00.pdf}), and he also discusses how using an ensemble can reduce the errors from the following three issues.

\begin{itemize}
  \setlength{\itemindent}{1in}
  \item[\textbf{Statistical:}] A lack of data may allow multiple models to fit the training set well.
  \item[\textbf{Computational:}] Optimization procedures occasionally converge to local solutions instead of the global solution.
  \item[\textbf{Representational:}] It may not be possible to model the true phenomenon with a known model.
\end{itemize}

\begin{figure}[!ht]
  \centering
  \includegraphics[scale = 0.7]{dietterich.png}
\end{figure}

The statistical problem refers to the lack of data to support a
particular hypothesis. The problem can be formulated as finding the
best hypothesis among competing models in the space
$\mathbf{\mathcal{H}}$. In the top left graph of the
depiction from Dietterich we see a blue boundary, and the idea is that all
models within this boundary will give the same fit to the training data.
Thus, there is insufficient information to determine which one is better.
By combining the models, we reduce the risk of choosing a terrible model.
For example, if we only observe two data points for a country, then
fitting a linear line or a log curve can both give the same accuracy
on the training data and we may have no information to distinguish
between the two.\\

The second problem is that some models are fit by optimizing some cost function.  These numerical algorithms can often converge to local solutions instead of the global solution.  The top right graph from Dietterich represents this problem, with points h1, h2, and h3 representing the local solutions and f the true global solution.  Thus, combining the multiple fits should get us closer to the true optimum f.  At the time of writing this vignette, no models which use this numerical optimization are present in the default methodology; however, we could introduce such models in the future (for example, a neural network).

The final problem, representational, refers to the fact that the true
function $f$ can not be represented by any of the individual models. However, by
combining the models we may expand the space of representable
functions and more closely approximate the true function $f$.  For example, if
the production of a country has been growing at a linear rate in the
distant past but has expanded rapidly recently, then neither a linear or
exponential model will provide a satisfactory result. However, an
ensemble combining a linear and exponential model will provide a better
solution by capturing different characteristics of the data.\\

From an implementation point of view, the algorithm is adaptive and will not need constant updating.  For example, If the data generating mechanism changes in the future, the next fit of the ensemble will shift weights to models which better represent the data and thus it will not be necessary to constantly monitor and update the methodologies/models manually.

\subsection{Description of Models}

This section describes the different base learners for the ensemble methodology, and they are listed in increasing order of complexity. An effective ensemble will have base models as diverse as possible. If there is no diversity and all models generate similar results, then little is gained by combining these models and the ensemble model will not be much of an improvement from an individual model.\\

\begin{itemize}
  \setlength{\leftmargini}{5em}
  \item [Mean:] Mean of all observations
  \item [Linear:] Linear Regression
  \item [Exponential:] Exponential function
  \item [Logistic:] Logistic function        
  \item [Naive:] Linear interpolation followed by last observation
    carried forward and first observation carried backward.
  \item [ARIMA:] Autoregressive Integrated Moving Average model
    selected based on the AICC, and imputation via Kalman Filter.
  \item [LOESS:] Local regression with linear models and model window
    varying based on sample size.
  \item [Splines:] Cubic spline interpolation.
  \item [MARS:] Multivariate Adaptive Regression Spline
\end{itemize}

\subsection{Extrapolation}

Describe the purpose and show examples of extrapolation weights.

\subsection{Computation of Weights}

Describe the leave-one-out cross-validation procedure.

\section{Case Studies}

Show a bunch of examples where certain products were imputed.

% \section{Simulation Study}
% 
% If this is a priority, do a simulation study to show that the method of
% imputation via ensembles is effective.

\end{document}
